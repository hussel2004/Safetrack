\section{Phase de Conception}
\label{sec:conception}

Cette phase détaille l'architecture et les choix de conception qui structurent la solution SafeTrack.

\subsection{Architecture logicielle globale}
Le système SafeTrack repose sur une architecture \textbf{n-tiers} modulaire, assurant une séparation claire des responsabilités entre la collecte de données, le traitement métier et la présentation utilisateur.

Le flux de données global est le suivant :
\begin{enumerate}
    \item \textbf{Capteurs IoT (GPS)} : Les modules SIM808 embarqués dans les véhicules capturent la position (latitude, longitude, vitesse) et l'envoient via le réseau LoRaWAN.
    \item \textbf{Gateway & Network Server (ChirpStack)} : Les paquets LoRa sont reçus par la passerelle et transmis à ChirpStack, qui gère le décodage initial et l'authentification des dispositifs.
    \item \textbf{Serveur d'Application (Backend)} : Notre API FastAPI reçoit les données via Webhook, effectue le parsing applicatif, vérifie les règles de geofencing et stocke les informations.
    \item \textbf{Clients (Frontend)} : L'application mobile Flutter interroge l'API pour afficher la position des véhicules en temps réel sur une carte interactive.
\end{enumerate}

Le diagramme de contexte et les modèles d'analyse sont détaillés dans la section \textbf{17.10.8 Modélisation UML - Analyse}.

\subsection{Architecture client-serveur}
L'architecture client-serveur est basée sur le protocole \textbf{HTTP/REST}, privilégiant une communication asynchrone et sans état (stateless).
\begin{itemize}
    \item \textbf{Serveur (API)} : Expose des ressources (/vehicles, /positions, /geofences) manipulables via les verbes standards (GET, POST, PUT, DELETE). Il renvoie les données au format \textbf{JSON}.
    \item \textbf{Client (Mobile)} : Consomme ces API de manière asynchrone pour ne pas bloquer l'interface utilisateur (Non-blocking UI), garantissant une expérience fluide même en cas de latence réseau.
\end{itemize}

\subsection{Choix technologiques et justification}
Les technologies suivantes ont été sélectionnées pour leur performance et leur adéquation aux besoins du temps réel :

\subsubsection{Frontend : Flutter (Dart)}
Le framework Flutter a été choisi pour sa capacité à produire une application native pour Android et iOS à partir d'une base de code unique.
\begin{itemize}
    \item \textbf{Performance} : Le moteur de rendu Skia assure des animations fluides (60fps), cruciales pour la navigation cartographique.
    \item \textbf{Écosystème} : Utilisation de \texttt{flutter\_map} pour la cartographie (alternative open-source légère à Google Maps) et \texttt{provider} pour une gestion d'état efficace.
\end{itemize}

\subsubsection{Backend : FastAPI (Python)}
FastAPI s'est imposé face à Django ou Flask pour plusieurs raisons :
\begin{itemize}
    \item \textbf{Asynchronisme} : Support natif de \texttt{async/await}, essentiel pour gérer de multiples connexions IoT simultanées sans bloquer le serveur.
    \item \textbf{Robustesse} : Validation automatique des données via \textbf{Pydantic}, réduisant les erreurs de type à l'exécution.
    \item \textbf{Documentation} : Génération automatique de la documentation Swagger/OpenAPI.
\end{itemize}

\subsubsection{Base de Données : PostgreSQL}
PostgreSQL a été retenu pour sa fiabilité et ses fonctionnalités avancées :
\begin{itemize}
    \item \textbf{Support JSONB} : Permet de stocker les payloads bruts des capteurs IoT de manière flexible.
    \item \textbf{Extensions Géographiques} : Bien que nous ayons implémenté des calculs haversine personnalisés (SQL), PostgreSQL est prêt pour PostGIS si les besoins évoluent.
\end{itemize}

\subsection{Conception fonctionnelle}
Les fonctionnalités clés ont été modélisées pour répondre aux cas d'utilisation critiques :
\begin{itemize}
    \item \textbf{Tracking Temps Réel} : Réception des coordonnées, calcul de vitesse instantanée et mise à jour de l'état du véhicule (En mouvement / Arrêt).
    \item \textbf{Geofencing} : Algorithme de détection de présence dans une zone sécurisée circulaire. Le système vérifie à chaque nouvelle position si la distance au centre de la zone est inférieure au rayon défini, grâce à la fonction SQL optimisée \texttt{est\_dans\_zone}.
    \item \textbf{Système d'Alertes} : Génération automatique d'alertes (HORS\_ZONE, VITESSE\_EXCESSIVE) basées sur des règles métier strictes.
\end{itemize}

\subsection{Conception de la base de données}
Le schéma relationnel s'articule autour des entités suivantes :
\begin{itemize}
    \item \textbf{vehicule} : Contient les identifiants techniques LoRaWAN (\texttt{deveui}, \texttt{appkey}) et l'état courant.
    \item \textbf{position\_gps} : Table historique stockant chaque relevé avec son timestamp. Elle est fortement indexée pour optimiser les requêtes temporelles.
    \item \textbf{zone\_securisee} : Définit les geofences (latitude, longitude, rayon).
    \item \textbf{alerte} : Enregistre les événements critiques pour l'audit.
\end{itemize}

\begin{figure}[H]
    \centering
    \includegraphics[width=0.9\textwidth]{images/schema_bdd1.png}
    \caption{Schéma Relationnel de la Base de Données - Partie 1 (Véhicules et Positions).}
    \label{fig:database_schema_part1}
\end{figure}

\begin{figure}[H]
    \centering
    \includegraphics[width=0.9\textwidth]{images/schema_bdd2.png}
    \caption{Schéma Relationnel de la Base de Données - Partie 2 (Zones et Alertes).}
    \label{fig:database_schema_part2}
\end{figure}

Des \textbf{triggers SQL} (ex: \texttt{trg\_create\_alerte\_hors\_zone}) ont été mis en place pour automatiser la détection d'anomalies directement au niveau de la base de données, garantissant une réactivité immédiate.

\subsection{Maquettage et conception de l'interface utilisateur}
L'interface utilisateur suit les directives du \textbf{Material Design 3} avec une identité visuelle personnalisée ("Deep Navy" et "Sky Blue"). L'ergonomie privilégie une navigation simplifiée et un feedback visuel immédiat pour les conditions de mobilité.

\begin{figure}[H]
    \centering
    \includegraphics[width=0.3\textwidth]{images/screen_login.jpg}
    \includegraphics[width=0.3\textwidth]{images/screen_map.jpg}
    \includegraphics[width=0.3\textwidth]{images/screen_geofence.jpg}
    \caption{Maquettes de l'interface : Écran de connexion, Carte de suivi et Gestion de zone.}
    \label{fig:ui_screens}
\end{figure}
