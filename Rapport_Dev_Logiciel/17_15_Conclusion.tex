\section{Conclusion du développement logiciel}
\label{sec:conclusion}

Le projet SafeTrack aboutit à une solution fonctionnelle et robuste, répondant aux exigences initiales de traçabilité et de sécurité des véhicules.

Ce développement a permis de :
\begin{itemize}
    \item \textbf{Maîtriser une architecture complète} : De l'objet connecté (IoT) jusqu'au cloud et à l'application mobile.
    \item \textbf{Valider les choix technologiques} : La combinaison Flutter + FastAPI + PostgreSQL s'est révélée performante et adaptée aux contraintes du temps réel.
    \item \textbf{Livrer un produit fiable} : Grâce à une stratégie de tests rigoureuse et une intégration continue.
\end{itemize}

Pour les prochaines versions, plusieurs évolutions sont envisageables :
\begin{itemize}
    \item Intégration de \textbf{WebSockets} pour remplacer le polling HTTP et réduire encore la latence.
    \item Ajout de \textbf{Notifications Push (FCM)} pour alerter l'utilisateur même application fermée.
    \item Développement d'un module d'analyse prédictive pour anticiper les maintenances véhicules basées sur l'historique des données télémétriques.
\end{itemize}
