\section{Phase de Tests et Validation}
\label{sec:tests}

Cette phase cruciale assure la fiabilité du système avant son déploiement final.

\subsection{Stratégie de tests logiciels}
Nous avons adopté la stratégie de la "Pyramide de Tests", favorisant une base large de tests unitaires et d'intégration, complétée par des tests système et d'interface.

\subsection{Tests Unitaires}
Les tests unitaires se focalisent sur la logique métier critique du Backend, isolée de la base de données réelle (Mocking). Ils sont principalement regroupés dans le fichier \texttt{test\_unitaires.py}.
\begin{itemize}
    \item \textbf{Calculs Géographiques} : Vérification que la fonction de distance Haversine renvoie des valeurs correctes à ±1 mètre près.
    \item \textbf{Parsing de Trames} : Validation que le parser extrait correctement les données d'une chaîne hexadécimale, et rejette les formats invalides.
\end{itemize}

\subsection{Tests d'Intégration}
Les tests d'intégration valident la communication entre l'API et la base de données PostgreSQL.
\begin{enumerate}
    \item \textbf{Scénario "Happy Path"} : Création d'un véhicule -> Ajout d'une position -> Récupération de l'historique. Le test valide que les données sont persistées et ressorties intègres.
    \item \textbf{Scénario d'Erreur} : Tentative de création d'un véhicule avec un \texttt{deveui} dupliqué. Le test confirme que l'API renvoie bien une erreur 409 Conflict.
\end{enumerate}

\subsection{Tests Fonctionnels}
Ces tests simulent des cas d'utilisation réels grâce à des scripts Python dédiés :
\begin{itemize}
    \item \textbf{Script} : \texttt{simulate\_geofence\_lifecycle.py}
    \item \textbf{Objectif} : Simuler le trajet complet d'un véhicule entrant et sortant d'une zone sécurisée.
    \item \textbf{Résultat Attendu} : Le système doit générer une alerte "HORS\_ZONE" précise dès que la position simulée dépasse le rayon défini.
\end{itemize}

\subsection{Tests d'Interface Utilisateur}
Tests manuels effectués sur des terminaux Android physiques et émulés :
\begin{itemize}
    \item \textbf{Réactivité de la Carte} : Fluidité du zoom et du déplacement (Pan/Zoom) avec plusieurs marqueurs.
    \item \textbf{Affichage des Alertes} : Vérification que les notifications visuelles (changement de couleur, pop-up) apparaissent bien lors d'un événement critique.
\end{itemize}

\subsection{Tests de Performance}
Nous avons testé la tenue en charge du système :
\begin{itemize}
    \item \textbf{Insertion Massive} : Injection de 100 positions par seconde via un script de charge.
    \item \textbf{Optimisation} : Ajout d'index SQL sur les colonnes \texttt{deveui} et \texttt{timestamp} pour maintenir des temps de réponse API sous les 100ms.
\end{itemize}

\subsection{Tests de Sécurité}
\begin{itemize}
    \item \textbf{Injection SQL} : Les requêtes passent systématiquement par l'ORM SQLAlchemy ou des requêtes paramétrées, neutralisant les tentatives d'injection.
    \item \textbf{Mots de Passe} : Vérification que les mots de passe utilisateurs sont bien stockés sous forme de hash (Bcrypt) et jamais en clair.
\end{itemize}

\subsection{Résultats des tests et analyse}
La phase de test a permis de valider 100\% des fonctionnalités critiques (Tracking, Alerting). Les quelques régressions détectées (latence d'affichage initiale) ont été corrigées avant la livraison.

\begin{figure}[H]
    \centering
    \includegraphics[width=0.8\textwidth]{images/test_results.png}
    \caption{Capture d'écran de l'exécution réussie des tests unitaires et d'intégration.}
    \label{fig:test_results}
\end{figure}
